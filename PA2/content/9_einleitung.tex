%%%%%%%%%%%%%%%%%%%%%%%%%%%%%%%%%%%%%%%%%%%%%%%%%%%%%%%%%%%%%%%%%%%%%%%%%%%%%%%
% Titel:   Einleitung
% Autor:   
% Datum:   13.12.2013
% Version: 0.0.0
%%%%%%%%%%%%%%%%%%%%%%%%%%%%%%%%%%%%%%%%%%%%%%%%%%%%%%%%%%%%%%%%%%%%%%%%%%%%%%%
%
%:::Change-Log:::
% Versionierung erfolgt auf folgende Gegebenheiten: -1. Release Versionen
%                                                   -2. Neue Kapitel
%                                                   -3. Fehlerkorrekturen
%
% 0.0.0       Erstellung der Datei
%%%%%%%%%%%%%%%%%%%%%%%%%%%%%%%%%%%%%%%%%%%%%%%%%%%%%%%%%%%%%%%%%%%%%%%%%%%%%%%
\chapter{Einleitung}\label{ch:einleitung_pa2}
	%
	Die \acrfull{ac:pa2} kn�pft l�ckenlos an die Arbeit aus der \acrfull{ac:pa1} an. Mit dem Abschluss der \gls{ac:pa2} sollen zwei Roboter bereitstehen, um an den nationalen Ausscheidungen in Burgdorf erfolgreich zu punkten. Die Problemstellung und Projektstrategie bleiben f�r die \gls{ac:pa2} weiter bestehen und sind in der (Dokumentation \gls{ac:pa1}) ersichtlich.\par 
	%
	Die ersten Tests an den mechanischen Systemen des kleinen Roboters haben bereits einige M�ngel aufgedeckt. Diese gilt es so rasch als m�glich zu bereinigen, damit der Roboter in Betrieb genommen werden kann. Dazu kommt das Entwickeln des zweiten Roboters, der die Aufgaben Catching the mammoths und Fire conquest erf�llen soll. Dabei soll die Mechanik wiederum m�glichst einfach und solide konstruiert werden. Ebenfalls soll die Anzahl der n�tigen Aktoren auf ein Minimum reduziert werden.\par
	%
	F�r die beiden Roboter ist eine Spielstrategie zu entwickeln. Diese sollte den Roboter erm�glichen auf die Geschehnisse w�hrend des Spiels zu reagieren. Dazu m�ssen alle Peripherien in einem Gesamtsystem zusammen funktionieren. \todo{gross10}
