%%%%%%%%%%%%%%%%%%%%%%%%%%%%%%%%%%%%%%%%%%%%%%%%%%%%%%%%%%%%%%%%%%%%%%%%%%%%%%%
% Titel:   Projektmanagement
% Autor:   meert1
% Datum:   05.01.2014
% Version: 0.2.0
%%%%%%%%%%%%%%%%%%%%%%%%%%%%%%%%%%%%%%%%%%%%%%%%%%%%%%%%%%%%%%%%%%%%%%%%%%%%%%%
%
%:::Change-Log:::
% Versionierung erfolgt auf folgende Gegebenheiten: -1. Release Versionen
%                                                   -2. Neue Kapitel
%                                                   -3. Fehlerkorrekturen
%
% 0.2.0       Erstellung der Kapitel und Unterkapitel
% 0.0.1       Erstellung der Datei
%%%%%%%%%%%%%%%%%%%%%%%%%%%%%%%%%%%%%%%%%%%%%%%%%%%%%%%%%%%%%%%%%%%%%%%%%%%%%%%    
\chapter{Projektmanagement PA2}\label{ch:projektmanagement_pa2}
	Das Projektmanagement der \gls{ac:pa2} wird Grunds�tzlich wie in der \gls{ac:pa1} gehandhabt (siehe Dokumentation \gls{ac:pa1}). Alles was nicht explizit erw�hnt wird, bleibt unver�ndert.
	%
	\section{Ressourcen}\label{s:ressourcen_pa2}
	   In der \gls{ac:pa1} sind einige Probleme betreffend dem vorankommen des Antriebs aufgetreten. Aus diesem Grund wurde eine Umstrukturierung der Teilgruppen geplant. Siehe Umstrukturierung in der Dokumentation der \gls{ac:pa1}. Diese wurde gr�sstenteils wie geplant umgesetzt. Einzig Herrn Kilian Balz, der zu je 50\% dem Kernteam und dem Antrieb zugeteilt wurde, ist in der \gls{ac:pa2} nicht mehr dabei. Die definitiven Einteilungen sind in der Tabelle \ref{tab:gesamtteam_pa2} zu finden. 
	%
	  \begin{table}[htbp]
	      \centering
	      \begin{tabular}{|l|l|l|l|l|l|} 
	          \hline
	          \rowcolor{bfhblue}
	          \textcolor{white}{Name} & \textcolor{white}{Vorname} & \textcolor{white}{K�rzel} & \textcolor{white}{E-Mail} & \textcolor{white}{Klasse} & \textcolor{white}{Team}\\
	          \hline
	          Grossenbacher & Simon & gross10 & \href{mailto: gross10@bfh.ch}{gross10@bfh.ch} & E3b & Kernteam\\
	          \hline
	          Meerstetter & Tobias & meert1 & \href{mailto: meert1@bfh.ch}{meert1@bfh.ch} & E3b & Antrieb\\
	          \hline
	          Rohrbach & Patrick & rohrp1 & \href{mailto: rohrp1@bfh.ch}{rohrp1@bfh.ch} & M3a & Kernteam\\
	          \hline
	          Roth & Hannes & rothh3 & \href{mailto: rothh3@bfh.ch}{rothh3@bfh.ch} & M3a & Kernteam\\
	          \hline
	          Haldemann & Jascha & haldj3 & \href{mailto: haldj3@bfh.ch}{haldj3@bfh.ch} & E3a & Kernteam\\
	          \hline
	          Heimsch & Gunnar & heimg1 & \href{mailto: heimg1@bfh.ch}{heimg1@bfh.ch} & E3b & Navigation\\
	          \hline
	          Plattner & Simon & plats1 & \href{mailto: plats1@bfh.ch}{plats1@bfh.ch} & E3a & Antrieb\\
	          \hline
	          Zurschmiede & Reto & zursr1 & \href{mailto: zursr1@bfh.ch}{zursr1@bfh.ch} & E3a & Navigation\\
	          \hline
	          K�ser & Nicola & kasen1 & \href{mailto: kasen1@bfh.ch}{kasen1@bfh.ch} & E3a & Naherkennung\&Kernteam\\
	          \hline
	          Greiler & Roland & greir1 & \href{mailto: greir1@bfh.ch}{greir1@bfh.ch} & M3a & Verdrahtung-/Volumen\\
	          \hline
	          Reust & Ralph & reusr6 & \href{mailto: reusr6@bfh.ch}{reusr6@bfh.ch} & M3a & Verdrahtung-/Volumen\\
	          \hline
	      \end{tabular}
	      \caption{�bersicht Eurobotteam PA2}
	      \label{tab:gesamtteam_pa2}  
	  \end{table}
	  %
	%
	\section{Projektablauf}\label{s:projektablauf_pa2}
		Der Ablauf ist der \cref{abb:pa2_ablauf} zu entnehmen.
		%
		\image{content/image/10_ablaufdiagramm_PA2}{scale=1.0}{htbp}[Projektablauf \gls{ac:pa2}][abb:pa2_ablauf]
	%
	\section{Zeitplan}
		\todo{Anpassen}
	%
	\section{Weiteres Vorgehen}\label{s:weiteres_vorgehen_pa2}
		Die \gls{ac:pa2} geht am 05.05.2014 zu ende. Bis zum Wettbewerb am 23.	 und 24.05.2014 gibt es aber noch einiges abzuarbeiten.
		Die beiden Roboter sind bis zum Stand der Dokumentation fertig montiert und verkabelt. Die Inbetriebnahme des kleinen Roboters ist in vollem Gange, muss aber noch fertiggestellt werden. Der grosse Roboter muss noch komplett in Betrieb genommen werden. Sobald die ersten Systeme zum Laufen kommen, m�ssen diese mit den vorbereiteten Testprotokollen getestet werden. 
		Des Weiteren wird von der Eurobot-Organisation ein kurzer Film, welcher das Team vorstellt, erwartet. Dieser ist in Planung, muss aber noch produziert werden.
		S�mtliche Vorbereitungen f�r den Wettbewerb wie organisieren und zusammenstellen von Reservematerial, Check-Listen und Werkzeugen stehen auch noch bevor.
