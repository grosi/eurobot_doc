%%%%%%%%%%%%%%%%%%%%%%%%%%%%%%%%%%%%%%%%%%%%%%%%%%%%%%%%%%%%%%%%%%%%%%%%%%%%%%%
% Titel:   Projektmanagement
% Autor:   meert1
% Datum:   05.01.2014
% Version: 0.2.0
%%%%%%%%%%%%%%%%%%%%%%%%%%%%%%%%%%%%%%%%%%%%%%%%%%%%%%%%%%%%%%%%%%%%%%%%%%%%%%%
%
%:::Change-Log:::
% Versionierung erfolgt auf folgende Gegebenheiten: -1. Release Versionen
%                                                   -2. Neue Kapitel
%                                                   -3. Fehlerkorrekturen
%
% 0.2.0       Erstellung der Kapitel und Unterkapitel
% 0.0.1       Erstellung der Datei
%%%%%%%%%%%%%%%%%%%%%%%%%%%%%%%%%%%%%%%%%%%%%%%%%%%%%%%%%%%%%%%%%%%%%%%%%%%%%%%    
\chapter{Projektmanagement PA2}\label{ch:projektmanagement_pa2}
	Das Projektmanagement der \gls{ac:pa2} wird Grunds�tzlich wie in der \gls{ac:pa1} gehandhabt (siehe Dokumentation \gls{ac:pa1}). Alles was nicht explizit erw�hnt wird, bleibt unver�ndert.
	%
	\section{Ressourcen}\label{s:ressourcen_pa2}
	   In der \gls{ac:pa1} sind einige Probleme betreffend dem vorankommen des Antriebs aufgetreten. Aus diesem Grund wurde eine Umstrukturierung der Teilgruppen geplant. Siehe Umstrukturierung in der Dokumentation der \gls{ac:pa1}. Diese wurde gr�sstenteils wie geplant umgesetzt. Einzig Herrn Kilian Balz, der zu je 50\% dem Kernteam und dem Antrieb zugeteilt wurde, ist in der \gls{ac:pa2} nicht mehr dabei. Die definitiven Einteilungen sind in der Tabelle \ref{tab:gesamtteam_pa2} zu finden. 
	%
	  \begin{table}[htbp]
	      \centering
	      \begin{tabular}{|l|l|l|l|l|l|} 
	          \hline
	          \rowcolor{bfhblue}
	          \textcolor{white}{Name} & \textcolor{white}{Vorname} & \textcolor{white}{K�rzel} & \textcolor{white}{E-Mail} & \textcolor{white}{Klasse} & \textcolor{white}{Team}\\
	          \hline
	          Grossenbacher & Simon & gross10 & \href{mailto: gross10@bfh.ch}{gross10@bfh.ch} & E3b & Kernteam\\
	          \hline
	          Meerstetter & Tobias & meert1 & \href{mailto: meert1@bfh.ch}{meert1@bfh.ch} & E3b & Antrieb\\
	          \hline
	          Rohrbach & Patrick & rohrp1 & \href{mailto: rohrp1@bfh.ch}{rohrp1@bfh.ch} & M3a & Kernteam\\
	          \hline
	          Roth & Hannes & rothh3 & \href{mailto: rothh3@bfh.ch}{rothh3@bfh.ch} & M3a & Kernteam\\
	          \hline
	          Haldemann & Jascha & haldj3 & \href{mailto: haldj3@bfh.ch}{haldj3@bfh.ch} & E3a & Kernteam\\
	          \hline
	          Heimsch & Gunnar & heimg1 & \href{mailto: heimg1@bfh.ch}{heimg1@bfh.ch} & E3b & Navigation\\
	          \hline
	          Plattner & Simon & plats1 & \href{mailto: plats1@bfh.ch}{plats1@bfh.ch} & E3a & Antrieb\\
	          \hline
	          Zurschmiede & Reto & zursr1 & \href{mailto: zursr1@bfh.ch}{zursr1@bfh.ch} & E3a & Navigation\\
	          \hline
	          K�ser & Nicola & kasen1 & \href{mailto: kasen1@bfh.ch}{kasen1@bfh.ch} & E3a & Naherkennung\&Kernteam\\
	          \hline
	          Greiler & Roland & greir1 & \href{mailto: greir1@bfh.ch}{greir1@bfh.ch} & M3a & Verdrahtung-/Volumen\\
	          \hline
	          Reust & Ralph & reusr6 & \href{mailto: reusr6@bfh.ch}{reusr6@bfh.ch} & M3a & Verdrahtung-/Volumen\\
	          \hline
	      \end{tabular}
	      \caption{�bersicht Eurobotteam PA2}
	      \label{tab:gesamtteam_pa2}  
	  \end{table}
	  %
	%
	\newpage
	\section{Projektablauf}\label{s:projektablauf_pa2}
		Der Ablauf ist der \cref{abb:pa2_ablauf} zu entnehmen.
		%
		\image{content/image/10_ablaufdiagramm_PA2}{scale=0.65}{htbp}[Projektablauf \gls{ac:pa2}][abb:pa2_ablauf]
	%
	\section{Zeit- und Meilensteinplanung}\label{s:zeitplanung_pa2}
		Die Meilensteinplanung in Abbildung \ref{abb:meilensteine_pa2} auf Seite \pageref{abb:meilensteine_pa2} wurde vom \gls{ac:kernteam} ganz zu Beginn der \gls{ac:pa2} erstellt und den anderen Teams kommuniziert. Um alle Teams anzuspornen, wurde in dieser Meilensteinplanung das Optimum bzw. Wunschdenken definiert. Es ist daher nicht verwunderlich, dass die Meilensteine f�r den kleinen und grossen Roboter nicht immer eingehalten werden konnten. N�hers ist in den jeweiligen Dokumentationen der einzelnen Teams beschrieben.
		\par
		Die Meilensteine die den mechanischen Teil des \gls{ac:kernteam}s betreffen (grau) konnten mit jeweils einer Verschiebung von einer bis zwei Wochen erreicht werden. Dank gen�gend eingeplanter Buffer-Zeit stellte dies jedoch kein Problem dar.
		%
%		\todo{grosi: bilder optimal skalieren (danke im voraus)}
		\begin{rotpdf90}
			\image{content/image/10_meilensteine_PA2.PNG}{scale=0.6, angle=90}{htbp}[Meilensteine PA2][abb:meilensteine_pa2]
		\end{rotpdf90}
		\par
		Der Zeitplan in Abbildung \ref{abb:zeitplan_pa2} auf Seite \pageref{abb:zeitplan_pa2} beschreibt den Verlauf der \gls{ac:pa2} von Seiten der Elektronik. Sie ist nach den Teammitgliedern unterteilt:		
		%
		\begin{rotpdf90}
			\image{content/image/10_zeitplanung_PA2.PNG}{scale=0.6, angle=90}{htbp}[Zeitplanung PA2][abb:zeitplan_pa2]	
		\end{rotpdf90}
		\par
		%
		\subsubsection{Ist-Aufnahme}\label{sss:istaufnahme_pa2}
		Grunds�tzlich wurden die meisten Arbeiten p�nktlich fertig gestellt. Bei der Bearbeitung der Bedieneinheit (Jaschas-Teil) hat sich gezeigt, dass zu viel Zeit f�r die Implementierung der SW und zu wenig f�r das Testen\&Verbessern eingeplant wurde. Schlussendlich hat sich dies jedoch ausgeglichen.\\
		Aufgrund einigen Praktika in den Vertiefungsmodulen die beim Zeitpunkt der Zeitplanung noch nicht bekannt waren, hat sich die urspr�nglich eingeplante Buffer-Zeit mehrheitlich aufgel�st.
		\par
		Im Teil von Simon und Nicola haben sich gegen Ende der \gls{ac:pa2} gewisse Verz�gerungen von ca. einer Woche ergeben. Der Grund war eine teilweise versp�te Verkabelung der Roboter. Dies hatte Auswirkungen auf die Umsetzung der Spielstrategie, die im Kapitel \ref{s:speilstrategie} auf Seite \pageref{s:speilstrategie} n�her beschreiben ist. F�r diese w�re es grunds�tzlich so wie so optimaler gewesen mehr Zeit einzuplanen, was aufgrund der allgemein knappen Projektzeit und dem fr�hen Start der Bachelor Thesis jedoch kaum m�glich war.
	%
	\section{Weiteres Vorgehen}\label{s:weiteres_vorgehen_pa2}
		Die \gls{ac:pa2} geht am 05.05.2014 zu ende. Bis zum Wettbewerb am 23. und 24.05.2014 gibt es jedoch noch einiges abzuarbeiten.
		Die beiden Roboter sind bis zum Stand der Dokumentation fertig montiert und verkabelt. Die Inbetriebnahme des kleinen Roboters ist in vollem Gange, muss aber noch fertiggestellt werden. Der grosse Roboter muss noch komplett in Betrieb genommen werden. Sobald die ersten Systeme zum Laufen kommen, m�ssen diese mit den vorbereiteten Testprotokollen getestet werden. Darauf folgt der Feinschliff beim l�sen der Aufgaben und der Strategie.\\
		Des Weiteren wird von der Eurobot-Organisation ein kurzer Film, welcher das Team vorstellt, erwartet. Dieser ist in Planung, muss aber noch produziert werden.\\
		S�mtliche Vorbereitungen f�r den Wettbewerb wie organisieren und zusammenstellen von Reservematerial, Check-Listen und Werkzeugen stehen auch noch bevor.
