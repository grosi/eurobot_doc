%%%%%%%%%%%%%%%%%%%%%%%%%%%%%%%%%%%%%%%%%%%%%%%%%%%%%%%%%%%%%%%%%%%%%%%%%%%%%%%
% Titel:   Testkonzept
% Autor:   
% Datum:   15.01.2014
% Version: 0.0.2
%%%%%%%%%%%%%%%%%%%%%%%%%%%%%%%%%%%%%%%%%%%%%%%%%%%%%%%%%%%%%%%%%%%%%%%%%%%%%%%
%
%:::Change-Log:::
% Versionierung erfolgt auf folgende Gegebenheiten: -1. Release Versionen
%                                                   -2. Neue Kapitel
%                                                   -3. Fehlerkorrekturen
% 0.0.2       Text und Kapitel hinzugef�gt.
% 0.0.1       Erstellung der Datei
%%%%%%%%%%%%%%%%%%%%%%%%%%%%%%%%%%%%%%%%%%%%%%%%%%%%%%%%%%%%%%%%%%%%%%%%%%%%%%%    
\chapter{Testkonzept PA2}\label{ch:validierung_pa2}

	%
    %Review Test
    \section{Review Test}\label{s:review_test_pa2}
%    	Der Review Test wird f�r jedes Modul einzeln durchgef�hrt. Dabei soll darauf geachtet werden, dass nicht der Verantwortliche des jeweiligen Moduls den Review Test selber durchf�hrt. Anschliessend an den Test, werden die nicht erf�llten Punkte vom Modulzust�ndigen korrigiert. Der Review Test ist in vier Teile gegliedert:
%    	%
%   		\begin{description}
%   			\item[Kommentark�pfe] Beinhaltet alle Tests bez�glich der Header- und Funktionsk�pfe.
%   			%
%   			\item[Namensgebung] In diesem Abschnitt werden die Code-Richtlinien, die zu Beginn des Projekts festgelegt wurden.
%   			%
%   			\item[Kommentar] Hier wird der Kommentar auf die festgelegten Richtlinien getestet.
%   			%
%   			\item[Kommentarstil] Beim Kommentarstil wird einerseits darauf geachtet, ob der Kommentar sinnvoll ist und ob er verst�ndlich ist, andererseits wird auch die Formatierung noch ein wenig angeschaut.
%   		\end{description}
    %
    %Funktionstest
    \section{Funktionstest}\label{s:funktionstest_pa2}
%    	Im Gegensatz zum Review Test wurde nur ein Funktionstest f�r das gesamte Projekt entworfen. Der Funktionstest ist zum momentanen Zeitpunkt in vier Kategorien unterteilt:
%    	%
%   		\begin{description}
%   			\item[Bilderkleben] Beinhaltet alle Test, die zur erfolgreichen Ablauf des Bilderklebens n�tig sind.
%   			%
%   			\item[Ballspicken] Test die den reibungslosen Vorgang des Ballspickens gew�hrleisten sollen. 
%   			%
%   			\item[CAN Test] In diese Kategorie geh�ren alle Tests, die die Kommunikation zwischen den einzelnen Knoten im Roboter sicherstellen.
%   			%
%   			\item[Gesamttests] Am Ende wird ein Gesamttest durchgef�hrt um auch das Zusammenspiel der einzelnen Komponenten testen zu k�nnen.
%   		\end{description}
%    	%