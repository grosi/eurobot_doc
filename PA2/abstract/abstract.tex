%%%%%%%%%%%%%%%%%%%%%%%%%%%%%%%%%%%%%%%%%%%%%%%%%%%%%%%%%%%%%%%%%%%%%%%%%%%%%%%
% Titel:   Abstract
% Autor:   S. Grossenbacher
% Datum:   27.09.2013
% Version: 1.0.0
%%%%%%%%%%%%%%%%%%%%%%%%%%%%%%%%%%%%%%%%%%%%%%%%%%%%%%%%%%%%%%%%%%%%%%%%%%%%%%%

%:::Change-Log:::
% Versionierung erfolgt auf folgende Gegebenheiten: -1. Stelle Semester
%                                                   -2. Stelle neuer Inhalt
%                                                   -3. Fehlerkorrekturen
%
% 1.0.0       Erstellung der Datei

%%%%%%%%%%%%%%%%%%%%%%%%%%%%%%%%%%%%%%%%%%%%%%%%%%%%%%%%%%%%%%%%%%%%%%%%%%%%%%%
\chapter*{Abstract}
	Als \gls{ac:kernteam} besteht eine unsere Hauptaufgaben darin die vier Subteams zu koordinieren. Ganz zu Beginn der \gls{ac:pa2} muss in diesem Aufgabenbereich gehandelt werden. Nach Abschluss der \gls{ac:pa1} sind die Zielsetzungen des Subteams Antrieb nicht zufriedenstellend erreicht worden. Damit in der \gls{ac:pa2} die Defizite aufgeholt werden k�nnen ist eine personelle Umstrukturierung zwingend n�tig. Das Antriebsteam wird komplett neu gebildet. Zus�tzlich sind freigeworden personell Ressourcen in Subteams untergebracht worden, welche diese in der \gls{ac:pa2} ben�tigen. Das \gls{g:eurobot}-Team besteht in der \gls{ac:pa2} noch aus 11 Studenten.
	\par
	Die \gls{ac:pa2} befasst sich ansonsten prim�r mit dem Bereinigen der Mechanik im ersten Roboter und dem Erstellen des zweiten Roboters mit den Aufgaben \gls{g:fire} und \gls{g:mammut_catch} (Funny Action). Dazu kommt die Inbetriebnahme der beiden Roboter sowie das Testen der fertigen Teil- und Gesamtsysteme. Die Strategie welcher die beiden Roboter zu folgen haben wird aus der \gls{ac:pa1} �bernommen.
	\par 
	Konstruktive �nderungen mussten vor allem am Speerspick-Mechanismus vorgenommen werden. Erste Funktionstests des Systems zeigten, dass eine Vereinzelung der Speere unumg�nglich ist. Dazu kamen weitere kleine �nderungen im Bereich des Antriebs und des Linearmoduls der Bilderaufgabe. 
	Bei der Entwicklung der Handlingsysteme des zweiten Roboters wird vor allem die Umsetzung der Netzaufgabe in den Vordergrund gestellt. Dazu werden verschiedene Testaufbauten gebaut und getestet. Der Entscheid f�llt schlussendlich auf ein Wurfmechanismus �hnlich einer Armbrust, jedoch sind die Spannelemente am Netz befestigt. F�r die Feueraufgabe sind w�hrend der Konzeptphase keine Tests durchgef�hrt worden. Der Entscheid f�llt auf ein Hebelsystem welches �ber einen Vakuumsauger Feuer anheben und transportieren, jedoch nicht wenden kann.
	\par
	In der Ausarbeitungsphase stellten sich die Ausl�sung des Netzes f�r die \gls{g:mammut_catch}-Aufgabe, sowie das Netz selbst als grosse Herausforderung heraus. Schlussendlich wurde ein Ausl�semechanismus konstruiert, welcher �ber ein Servo ausgel�st, jedoch ohne Servobet�tigung geladen werden kann. Es werden verschiedene Netze mit unterschiedlichen Steifigkeiten und Maschengr�ssen getestet. Die Wahl des Netzes f�llt schliesslich auf ein handels�bliches Feumernetz, das noch modifiziert wird. Die Feueraufgabe ist �ber ein Vakuumsaugsystem gel�st. Dabei kann der Vakuumsauger vertikal bewegt werden. Die Bewegung wird �ber ein Servo generiert.
	\par 
	Die Erkenntnisse des Navigationsteams aus der \gls{ac:pa1} zeigen, dass die vorhandenen Systeme Laser und Ultraschallnavigation nicht weiter zu verfolgen sind. Als Alternative wird ein Ultraschall-System eingekauft. Das eingekaufte System kann die Anforderungen zufriedenstellend erreichen.
	\par
	�ber das ganze Eurobotteam gesehen werden die Zielsetzungen der \gls{ac:pa2} gut erreicht. Die Probleme mit dem Antrieb haben sich aber weit in die \gls{ac:pa2} hineingezogen, was zu Verz�gerungen f�r das ganze \gls{g:eurobot}-Team f�hrte. So konnte bis zum Stand der Dokumentation noch nicht beide Roboter in Betrieb genommen werden. Was jegliche Tests des Gesamtsystems unm�glich macht.
%	\par
%	Die beiden Roboter sind in den Abbildungen \ref{abb:robo_gross} und \ref{abb:robo_klein} zu sehen.
	%
%	\image{}{scale=1}{htbp}[Grosse Roboter ("B52")][abb:robo_gross]
%	%
%	\image{}{scale=1}{htbp}[Kleine Roboter ("Ballerina")][abb:robo_klein]
	\begin{figure}[htbp] %htbp
		\centering
		\begin{subfigure}[b]{0.49\textwidth}
			\centering
			\includegraphics[height=8cm]{abstract/image/Roboter_klein}       
			\caption{"Ballerina"}
			\label{abb:robo_klein}
		\end{subfigure}
		\begin{subfigure}[b]{0.49\textwidth}
			\centering
			\includegraphics[height=8cm]{abstract/image/Roboter_gross}    
			\caption{"B52"}    
			\label{abb:robo_gross} 
		\end{subfigure}
		%\caption{Knoten ID's}
		%\label{abb:node_id}
   	\end{figure}