%%%%%%%%%%%%%%%%%%%%%%%%%%%%%%%%%%%%%%%%%%%%%%%%%%%%%%%%%%%%%%%%%%%%%%%%%%%%%%%
% Titel:   Schlusswort
% Autor:   
% Datum:   22.09.2013
% Version: 0.0.1
%%%%%%%%%%%%%%%%%%%%%%%%%%%%%%%%%%%%%%%%%%%%%%%%%%%%%%%%%%%%%%%%%%%%%%%%%%%%%%%
%
%:::Change-Log:::
% Versionierung erfolgt auf folgende Gegebenheiten: -1. Release Versionen
%                                                   -2. Neue Kapitel
%                                                   -3. Fehlerkorrekturen
%
% 0.0.1       Erstellung der Datei
%%%%%%%%%%%%%%%%%%%%%%%%%%%%%%%%%%%%%%%%%%%%%%%%%%%%%%%%%%%%%%%%%%%%%%%%%%%%%%%    
\chapter{Schlusswort}\label{ch:schlusswort}
	Den ersten Schritt zum Erfolg sehen wir in der Kommunikation. Bei einer abteilungs�bergreifenden Arbeit ist es wichtig st�ndig untereinander zu Kommunizieren, vor allem wir als \acrfull{ac:kernteam}, da wir den �berblick �ber das ganze Projekt haben sollten. Komplikationen m�ssen abgesprochen und diskutiert werden. Es wurde mehrmals festgestellt, dass L�sungsans�tze durch die Kommunikation untereinander einfacher erfolgten. Obwohl wir immer bestrebt waren m�glichst oft untereinander zu kommunizieren, mussten wir feststellen, dass der Informationsfluss nicht immer optimal war. In der \gls{ac:pa2} ist dies zu verbessern, da sicherlich vermehrt Entscheidungen getroffen werden und deshalb der Informationsfluss noch an Wichtigkeit zunimmt.
	%
	\paragraph{Mechanik} Wir haben eine grosse Reihe an Tests durchgef�hrt, um die besten Systeme f�r die einzelnen Aufgaben zu evaluieren. Wir sind �berzeugt, dadurch viel Zeit mit Korrekturen eingespart zu haben.
	%
	Bei den Fertigungszeichnungen wurden mehrmals zu wenige Toleranzen angegeben. Wir haben festgestellt, dass die Werkstatt nach dem korrekten Motto ?so genau wie n�tig? arbeitet. Diese Erkenntnis muss unbedingt in die \gls{ac:pa2} mitgenommen werden. Durch den zeitlichen Verzug, den wir in der Ausarbeitungsphase erhalten haben, wurde auch mehr Zeit f�r die Fertigung der Teile ben�tigt.
	%
	%
	\paragraph{Elektrotechnik} Die Analyse und Implementierung eines umfangreichen Kommunikationsprotokolls zu Beginn der Projektarbeit konnte rasch abgeschlossen und getestet werden. Auch wurde die grundlegenden Software-Strukturen f�r den sp�teren Verlauf des Projekts entworfen und umgesetzt. W�hrend dieser Phase trafen wir auf sehr fragw�rdig gestaltete Software-Module, f�r die einige Zeit in Analyse und Reimplementierung investiert wurde. Aufgrund der Verz�gerung bei der Mechanik konnte die geplante Peripherie-Ansteuerung/Konfiguration noch nicht umgesetzt werden, was nun in der \gls{ac:pa2} erledigt werden muss.\par
	%
	%
	\paragraph{R�sum�} Das \gls{g:eurobot}-Team hat das Hauptziel der \gls{ac:pa1}, in einem Spiel ohne Gegner einen Punkt erzielen zu k�nnen, nicht erreicht. Jedoch sind die Fortschritte innerhalb der Subteams sehr unterschiedlich. Innerhalb des Kernteams sind wir jedoch �berzeugt, dass die M�ngel erkennt wurden und die R�ckst�nde in der \gls{ac:pa2} aufgeholt werden k�nnen. Wir sind als \gls{g:eurobot}-Team �berzeugt, im Allgemeinen gut zu funktionieren und das jedes Mitglied weiterhin motiviert an einer erfolgreichen Teilnahme an den nationalen Ausscheidungen in Burgdorf arbeitet.