%%%%%%%%%%%%%%%%%%%%%%%%%%%%%%%%%%%%%%%%%%%%%%%%%%%%%%%%%%%%%%%%%%%%%%%%%%%%%%%
% Titel:   Testkonzept
% Autor:   
% Datum:   15.01.2014
% Version: 0.0.2
%%%%%%%%%%%%%%%%%%%%%%%%%%%%%%%%%%%%%%%%%%%%%%%%%%%%%%%%%%%%%%%%%%%%%%%%%%%%%%%
%
%:::Change-Log:::
% Versionierung erfolgt auf folgende Gegebenheiten: -1. Release Versionen
%                                                   -2. Neue Kapitel
%                                                   -3. Fehlerkorrekturen
% 0.0.2       Text und Kapitel hinzugef�gt.
% 0.0.1       Erstellung der Datei
%%%%%%%%%%%%%%%%%%%%%%%%%%%%%%%%%%%%%%%%%%%%%%%%%%%%%%%%%%%%%%%%%%%%%%%%%%%%%%%    
\chapter{Testkonzept}\label{ch:validierung}
	Die Testdurchl�ufe sind ein wichtiger Teil im Werdegang eines Projekts. Nur mit klaren, m�glichst alle F�lle abdeckenden und klar messbaren Testdurchl�ufen kann die Funktion sichergestellt werden. Es ist jedoch nicht m�glich alle Fehler in einem Projekt zu finden, sie k�nnen allerdings stark minimiert werden. Die Testphase wurde in zwei Teile unterteilt:
	%
	\begin{description}
		\item[Review Test] Beim Review Test wird eine optische Kontrolle des Codes durchgef�hrt, bei welchem unter anderem auf die festgelegten Code-Richtlinien oder auf die Qualit�t des Kommentars geachtet wird.
		%
		\item[Funktionstest] Der Funktionstest ist, wie der Name es bereits erahnen l�sst, f�r das Testen der Funktionen des Roboters zust�ndig. Es werden klare Ziele / Funktionen festgelegt, welche anschliessend getestet werden. Funktionstest sollen nicht nur Testf�lle beinhalten, von denen man erwartet, dass sie funktionieren, sondern es wird auch auf absichtlich herbeigef�hrte Fehler getestet.
	\end{description}
	%
	Auf Grund der bereits erw�hnten Verz�gerungen in der  \gls{ac:pa1} konnten die Test noch nicht durchgef�hrt werden. Die bereits definierten Tests sind im Anhang E \textit{Funktionstests Testprotokolle PA2} beigelegt.
	%
    %Review Test
    \section{Review Test}\label{s:review_test}
    	Der Review Test wird f�r jedes Modul einzeln durchgef�hrt. Dabei soll darauf geachtet werden, dass nicht der Verantwortliche des jeweiligen Moduls den Review Test selber durchf�hrt. Anschliessend an den Test, werden die nicht erf�llten Punkte vom Modulzust�ndigen korrigiert. Der Review Test ist in vier Teile gegliedert:
    	%
   		\begin{description}
   			\item[Kommentark�pfe] Beinhaltet alle Tests bez�glich der Header- und Funktionsk�pfe.
   			%
   			\item[Namensgebung] In diesem Abschnitt werden die Code-Richtlinien, die zu Beginn des Projekts festgelegt wurden.
   			%
   			\item[Kommentar] Hier wird der Kommentar auf die festgelegten Richtlinien getestet.
   			%
   			\item[Kommentarstil] Beim Kommentarstil wird einerseits darauf geachtet, ob der Kommentar sinnvoll ist und ob er verst�ndlich ist, andererseits wird auch die Formatierung noch ein wenig angeschaut.
   		\end{description}
    %
    %Funktionstest
    \section{Funktionstest}\label{s:funktionstest}
    	Im Gegensatz zum Review Test wurde nur ein Funktionstest f�r das gesamte Projekt entworfen. Der Funktionstest ist zum momentanen Zeitpunkt in vier Kategorien unterteilt:
    	%
   		\begin{description}
   			\item[Bilderkleben] Beinhaltet alle Test, die zur erfolgreichen Ablauf des Bilderklebens n�tig sind.
   			%
   			\item[Ballspicken] Test die den reibungslosen Vorgang des Ballspickens gew�hrleisten sollen. 
   			%
   			\item[CAN Test] In diese Kategorie geh�ren alle Tests, die die Kommunikation zwischen den einzelnen Knoten im Roboter sicherstellen.
   			%
   			\item[Gesamttests] Am Ende wird ein Gesamttest durchgef�hrt um auch das Zusammenspiel der einzelnen Komponenten testen zu k�nnen.
   		\end{description}
    	%