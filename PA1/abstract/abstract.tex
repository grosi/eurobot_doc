%%%%%%%%%%%%%%%%%%%%%%%%%%%%%%%%%%%%%%%%%%%%%%%%%%%%%%%%%%%%%%%%%%%%%%%%%%%%%%%
% Titel:   Abstract
% Autor:   S. Grossenbacher
% Datum:   27.09.2013
% Version: 1.0.0
%%%%%%%%%%%%%%%%%%%%%%%%%%%%%%%%%%%%%%%%%%%%%%%%%%%%%%%%%%%%%%%%%%%%%%%%%%%%%%%

%:::Change-Log:::
% Versionierung erfolgt auf folgende Gegebenheiten: -1. Stelle Semester
%                                                   -2. Stelle neuer Inhalt
%                                                   -3. Fehlerkorrekturen
%
% 1.0.0       Erstellung der Datei

%%%%%%%%%%%%%%%%%%%%%%%%%%%%%%%%%%%%%%%%%%%%%%%%%%%%%%%%%%%%%%%%%%%%%%%%%%%%%%%
\chapter*{Abstract}
	Im Rahmen der \gls{ac:pa1} der Studieng�nge Elektro- und Maschinentechnik an der \gls{ac:bfh} nehmen wir am \gls{g:eurobot}-Wettbewerb teil. Als \gls{ac:kernteam} bestanden unsere Hauptaufgaben darin, die vier Sub-Teams zu koordinieren, die Spielstrategie zu entwerfen, die Manipulationen des Spieles zu l�sen, das CAN-Bus/Protokoll zu definieren und ein Testkonzept f�r die einzelnen Module und das Gesamtsystem zu erstellen.\par 
	%
	Um das Vorankommen des gesamten \gls{g:eurobot}-Teams �berwachen zu k�nnen, wurde zu Beginn des Projekts f�r alle Teams ein sehr grober, aber klarer Zeitplan in Form von Meilensteinen zusammengestellt. Damit nicht zu viel Zeit verloren geht, setzten wir mit Tobias Meerstetter nur ein Teammitglied auf die �berwachung an. Er hatte �ber das ganze Team einen �berblick und wusste zu jedem Zeitpunkt wer mit welchen Problemen k�mpft.\par 
	%
	F�r die Strategiefindung wurde in einer ersten Phase entschieden Priorit�ten zu setzen und nicht alle m�glichen Aufgaben zu l�sen. Somit musste zuerst entschieden werden, was �berhaupt alles umgesetzt werden soll.
	Mittels einer Chancen Analyse wurden verschiedene L�sungsans�tze erarbeitet und schliesslich eine Zwei-Roboter-Strategie mit den Aufgaben \gls{g:fresko}, \gls{g:mammut}, \gls{g:fire} und \gls{g:mammut_catch} definiert. F�r die \gls{ac:pa1} galt das Ziel den ersten Roboter mit den Systemen f�r die Bilder- und die Mammutaufgabe hardwarem�ssig fertig zu Bauen.\par 
	%
	Um bestm�gliche Resultate zu erzielen wurden anschliessen auf der mechanischen Seite verschiedenste Testaufbauten ausgiebig getestet und evaluiert. Der Entscheid f�llt schlussendlich auf einen Spickmechanismus �ber eine Blattfeder f�r die Mammutaufgabe und auf ein Linearmodul f�r die Freskoaufgabe.\par 
	%
	In der Ausarbeitungsphase stellte sich auf mechanischer Seite das Spannen der Blattfeder als gr�sste Herausforderung heraus. Schlussendlich wurde ein Klinkensystem konstruiert, welches �ber ein einziges Servo gesteuert werden kann.\par 
	%
	Auf der Softwareseite wurde die einzelnen Module des kleinen Roboters verifiziert und definiert. Darauf aufbauend wurden m�gliche Kommunikationsm�glichkeiten analysiert und in die Bereiche Information (\acrfull{ac:elp}) und Steuerung unterteilt. F�r die beiden M�glichkeiten wurden Protokolle (\acrfull{ac:elp} und \acrfull{ac:goto}) basierend auf einer Master-Slave-Kommunikation erarbeitet.\par
	%
	\newpage
	Die Implementierung der Software wurde in mehrere Teile gegliedert. Zum einen wurde ein Applikationsbereich definiert, der \gls{g:rtos} bezogene Module beinhaltet. Die restliche Softwarebestandteile wurden dem Bereich Firmware zugeordnet. Er beinhaltet Code, der sich direkt auf die Peripherie des Mikrocontrollers bezieht. Zu den fertig und getesteten Implementierungen geh�rt die Kommunikation in Form eines Gatekeepers, grundlegende Steuertasks, sowie Servo-Ansteuerung, \gls{ac:swd} und \gls{ac:can}. Ausserdem wurde Code aus vergangenen \gls{g:eurobot}-Teams analysiert und verbessert.\par 
	%
	�ber das ganze \gls{g:eurobot}-Team gesehen werden die Zielsetzungen erreicht. Lediglich bei den Teams Antrieb und Navigation ergaben sich Schwierigkeiten. Bei der Navigation wird nun in der \gls{ac:pa2} ein neues eingekauftes Ultraschall-System zum Einsatz kommen.
	Das Antriebs-Team soll mit einer Umstrukturierung des Teams wider in eine zufriedenstellende Richtung gelenkt werden.
	%
	\image{content/image/1_kleiner_roboter}{scale=0.5}{htbp}[Resultat der \gls{ac:pa1}, der kleine Roboter][abb:kleiner_roboter]	
